\documentclass[11pt, oneside]{article}   	% use "amsart" instead of "article" for AMSLaTeX format
\usepackage{geometry}                		% See geometry.pdf to learn the layout options. There are lots.
\geometry{letterpaper}                   		% ... or a4paper or a5paper or ... 
%\geometry{landscape}                		% Activate for rotated page geometry
%\usepackage[parfill]{parskip}    		% Activate to begin paragraphs with an empty line rather than an indent
\usepackage{graphicx}				% Use pdf, png, jpg, or eps§ with pdflatex; use eps in DVI mode
								% TeX will automatically convert eps --> pdf in pdflatex	
\usepackage{amsmath}
%\usepackage{systeme}
\usepackage{amsthm}	
\usepackage{amssymb}
\newtheorem{proposition}{Proposition}
\newtheorem{identity}{Identity}

%SetFonts

%SetFonts


\title{Math Journal}
\author{Andrew Ha}
\date{}							% Activate to display a given date or no date

\begin{document}
\maketitle
\section*{Combinatorial Proof}
\begin{proposition} For positive integers $n$ and $k$ with $n=2k, \frac{n!}{2!^k}$ is an integer. \end{proposition}
\emph{Proof.}  Consider the $n$ symbols: $x_1, x_1, x_2, x_2, \cdots, x_k, x_k$.
The number of arrangements of all these $n = 2k$ symbols is an integer that equals
\[
\frac{n!}{\underbrace{2! 2! \cdots 2!}_{\text{k factors of 2!}}} = \frac{n!}{2!^k}
\]

I learned this from the first day of class in MACM 101. This is an example of proving that a value is an integer by obtaining that value from counting something. You can also use combinatorics in proof by double counting:
\begin{identity} 
\[
2^n = \sum_{k=0}^{n}  \binom{n}{k}
\]
\end{identity}

This can be proved by counting the number of subsets of a set with $n$ elements. 
%\subsection{}
\end{document}  